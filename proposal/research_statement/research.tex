% --------------- 12 POINT FONT -------------------------------
\documentclass[12pt]{article}
% --------------- 10 POINT FONT FOR CAPTIONS ------------------
\usepackage[font=footnotesize]{caption}
% --------------- NY TIMES FONT -------------------------------
\usepackage{times}
% --------------- 1 INCH MARGINS ------------------------------
\usepackage[margin=1in]{geometry}
% --------------- LINE SPACING --------------------------------
\usepackage{setspace}
\singlespacing
%\doublespacing
% --------------- SMALL SECTION TITLES ------------------------
\usepackage[tiny,compact]{titlesec}
% --------------- MATH PACKAGES -------------------------------
\usepackage{amsmath,amsthm,amssymb}
\begin{document}


% --------------- TITLE AND NAME ------------------------------
\begin{center}
Name
\hfill
Research Statement \\
\large{\bf Title} \\
\end{center}
% --------------- CONTENT -------------------------------------

\noindent
Keywords: \\
\newline

\noindent
\textbf{Evidence of Intellectual Merit can be found in all parts of the application:
Personal Statement, Research Plan, letters, experiences, awards, achievements, and transcripts.}

\noindent
\textbf{Note that your intellectual merit is different from your research’s intellectual merit}\\

\noindent
\textbf{Background:}
In recent years, there has been a publication, \cite{huang2018yolo}.

\noindent
\textbf{Bridge Background to Problem Statement}

\noindent
\textbf{Problem Statement:}

\noindent
\textbf{problem 1} Current limitations?\\

\noindent
\textbf{problem 2} Current scientific gaps?\\

\noindent
\textbf{problem 3} How can our contribution advance SOTA? \\

\noindent
\textbf{Intellectual Merit (not a section):} The Intellectual Merit criterion encompasses the potential to advance knowledge.\\
What is the potential for the proposed activity to:\\
Advance knowledge and understanding within its own field or across different fields (Intellectual Merit)?\\
Demonstrated intellectual ability (grades, curricula, awards, publications, presentations, etc.)
Other evidence of your potential, such as the ability to:\\
Plan and conduct research\\
Work as a member of a team as well as independently\\
Interpret and communicate research\\
Take initiative, solve problems, persist.\\

\noindent
\textbf{Research Plan:} To what extent do the proposed activities suggest and explore creative, original, or potentially transformative concepts?\\
Is the plan for carrying out the proposed activities well-reasoned, well-organized, and based on a sound rationale? Does the plan incorporate a mechanism to assess success?\\
\textbf{Aim 1} You can type your Aim 1 here if you have one. If you don't then you can delete the subsection\\
\textbf{Aim 2} You can type your Aim 1 here if you have one. If you don't then you can delete the subsection\\

\noindent
\textbf{WHAT IS THE RESEARCH? (not a section)}

\noindent
\textbf{Broader Impacts:} The Broader Impacts criterion encompasses the potential to benefit society and contribute to the achievement of specific, desired societal outcomes.\\
To promote the progress of science\\
To advance the national health, prosperity, and welfare\\
To secure the national defense\\
What benefit society or advance desired societal outcomes (Broader Impacts)?\\
To what extent do the proposed activities suggest and explore creative, original, or potentially transformative concepts?\\
Societal benefits may include, but are not limited to:\\
Increasing participation of underrepresented groups, women, persons with disabilities, veterans\\
Outreach: Mentoring; improving STEM education in schools\\
Increasing public scientific literacy; increased public engagement with STEM\\
Community outreach: science clubs, radio, TV, newspapers, blogs\\
Increasing collaboration between academia, industry, others\\

\noindent
\textbf{Evidence of Broader Impacts can be in all parts of the application: Personal Statement, Research Plan, letters, experiences, awards, achievements}\\

\noindent
\textbf{Qualifications (not a section):} How well qualified is the individual, team, or organization to conduct the proposed activities? \\
How did you learn about this field? E.g. through classes, readings, seminars, work or other experiences, or conversations with people already in the field?\\
How have you capitalized on opportunities available to you?\\
What reasons can you give for reviewers to be interested in your application?\\
What impact have you had on your academic, local and the broader community?\\

\noindent
\textbf{Available Resources (not a section):}
Are there adequate resources available to the PI (either at the home organization or through collaborations) to carry out the proposed activities?\\

% --------------- WORKS CITED (10pt FONT) ---------------------

\footnotesize
\bibliographystyle{ieee}
\bibliography{ref}
\end{document}

% -------------------------------------------------------------

% -------------------------------------------------------------
